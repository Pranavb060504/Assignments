\documentclass{beamer}
\usetheme{CambridgeUS}

\setbeamertemplate{caption}[numbered]{}

\usepackage{enumitem}
\usepackage{tfrupee}
\usepackage{amsmath}
\usepackage{amssymb}
\usepackage{gensymb}
\usepackage{graphicx}
\usepackage{txfonts}

\def\inputGnumericTable{}

\usepackage[latin1]{inputenc}                                 
\usepackage{color}                                            
\usepackage{array}                                            
\usepackage{longtable}                                        
\usepackage{calc}                                             
\usepackage{multirow}                                         
\usepackage{hhline}                                           
\usepackage{ifthen}
\usepackage{caption} 
\captionsetup[table]{skip=3pt}  
\providecommand{\pr}[1]{\ensuremath{\Pr\left(#1\right)}}
\providecommand{\cbrak}[1]{\ensuremath{\left\{#1\right\}}}
\renewcommand{\thefigure}{\arabic{table}}
\renewcommand{\thetable}{\arabic{table}}                                     
                               
\title{AI1110 \\ Assignment 5}
\author{Pranav B \\ AI21BTECH11023}
\date{18 May 2022}


\begin{document}
	% The title page
	\begin{frame}
		\titlepage
	\end{frame}
	
	% The table of contents
	\begin{frame}{Outline}
    		\tableofcontents
	\end{frame}
	
	% The question
	\section{Question}
	\begin{frame}{ Papoulis Chapter 2 example 2.17}
	If we toss a coin twice, we generate the four outcomes $hh,ht,th$, and $tt$ 
	\end{frame}
	% The solution
	\section{Solution}
	\begin{frame}{Solution}
	(a) To construct an experiment with these outcomes, it suffices to assign probabilities to its elementary events. With $a$and $b$ two positive numbers such that $a + b = 1$,
we assume that\\
\begin{table}[ht!]
\begin{tabular}{c c c}
$P(hh) = a^2$ & $P(ht) = P(th) = ab$ & $P(tt) = b^2$
\end{tabular}
\end{table}
These probabilities are consistent with the axioms because
\begin{align}
a^2+ab+ab +b^2 = (a+b)^2 = 1
\end{align}
In the experiment so constructed, the events\\
$H_1$ = {heads at first toss} = $(hh,ht)$\\
$H_2$ = {heads at second toss} = $(hh,th)$\\
The intersection $H_1$ $H_2$ of these two events consists of the single outcome $(hh)$
Hence\\
\begin{align}
P(H_1 H_2) = P(hh) = a^2 = P(H_1 )P(H_2)
\end{align}
This shows that the events $H_1$ and $H_2$ are independent.
\end{frame}
\begin{frame}
(b) The experiment in part (a) of this example can be specified in terms of the
probabilities $P(H_1) = P(H_2) = a$ of the events $H_1$ and $H_2$ and the information that these events are independent.\\
Indeed as we have shown the events $H_1'$ and $H_2$ and the events $H_1'$ and $H_2'$ are also independent. Furthermore,\\
\begin{table}[ht!]
\begin{tabular}{c c c c}
$H_1 H_2=(hh)$ & $H_1 H_2=(ht)$ & $H_1 H_2=(th)$ & $H_1 H_2=(tt)$
\end{tabular}
\end{table}
and $P(H_1') = 1 - P(H_1) = 1 - a$,$ P(H_2') = 1 - P(H_2) = 1 - a$. Hence
\begin{table}
\begin{tabular}{c c c c}
$P(hh) = a^2$ & $P(ht) = a(l - a)$ & $P(th) = (l - a)a$ & $P(tt) = (1 - a)^2$
\end{tabular}
\end{table}
	\end{frame}
\end{document}