\documentclass{beamer}
\usetheme{CambridgeUS}

\setbeamertemplate{caption}[numbered]{}

\usepackage{enumitem}
\usepackage{tfrupee}
\usepackage{amsmath}
\usepackage{amssymb}
\usepackage{gensymb}
\usepackage{graphicx}
\usepackage{txfonts}

\def\inputGnumericTable{}
\newcommand*{\permcomb}[4][0mu]{{{}^{#3}\mkern#1#2_{#4}}}
\newcommand*{\perm}[1][-3mu]{\permcomb[#1]{P}}
\newcommand*{\comb}[1][-1mu]{\permcomb[#1]{C}}
\usepackage[latin1]{inputenc}                                 
\usepackage{color}                                            
\usepackage{array}                                            
\usepackage{longtable}                                        
\usepackage{calc}                                             
\usepackage{multirow}                                         
\usepackage{hhline}                                           
\usepackage{ifthen}
\usepackage{caption} 
\captionsetup[table]{skip=3pt}  
\providecommand{\pr}[1]{\ensuremath{\Pr\left(#1\right)}}
\providecommand{\cbrak}[1]{\ensuremath{\left\{#1\right\}}}
\renewcommand{\thefigure}{\arabic{table}}
\renewcommand{\thetable}{\arabic{table}}                                     
                               
\title{AI1110 \\ Assignment 8}
\author{Pranav B \\ AI21BTECH11023}
\date{6 June 2022}


\begin{document}
	% The title page
	\begin{frame}
		\titlepage
	\end{frame}
	
	% The table of contents
	\begin{frame}{Outline}
    		\tableofcontents
	\end{frame}
	
	% The question
	\section{Question}
	\begin{frame}{ Papoulis Chapter 8 problem 8.21}
	The random variable $x$ has the truncated exponential density $f(x)=ce^{-c(x-x_{0})}U(x-x_0)$. Find the ML estimate $\hat{c}$ of $c$ in terms of the n samples of $x_i$ of $x$.
	\end{frame}
	\section{Solution}
	\begin{frame}{Solution}
	The joint density,\\
	$f(X,c)&=c^{n}e^{-cn(\bar{x}-x_{0})}$ and $x_{i}>x_{0}$
	has an interior maximum if,
	\begin{align}
	\frac{\partial{f(X,c)}}{\partial{c}}=0\\
	\implies nc^{n-1}e^{-cn(\bar{x}-x_{0})}+c^{n}e^{-cn(\bar{x}-x_{0})}(-n(\bar{x}-x_{0}))=0\\
	\implies n+c(-n)(\bar{x}-x_{0})=0\\
	\therefore \hat{c}=\frac{1}{\bar{x}-x_{0}}
	\end{align}
	\end{frame}
\end{document}