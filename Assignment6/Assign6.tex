\documentclass{beamer}
\usetheme{CambridgeUS}

\setbeamertemplate{caption}[numbered]{}

\usepackage{enumitem}
\usepackage{tfrupee}
\usepackage{amsmath}
\usepackage{amssymb}
\usepackage{gensymb}
\usepackage{graphicx}
\usepackage{txfonts}

\def\inputGnumericTable{}
\newcommand*{\permcomb}[4][0mu]{{{}^{#3}\mkern#1#2_{#4}}}
\newcommand*{\perm}[1][-3mu]{\permcomb[#1]{P}}
\newcommand*{\comb}[1][-1mu]{\permcomb[#1]{C}}
\usepackage[latin1]{inputenc}                                 
\usepackage{color}                                            
\usepackage{array}                                            
\usepackage{longtable}                                        
\usepackage{calc}                                             
\usepackage{multirow}                                         
\usepackage{hhline}                                           
\usepackage{ifthen}
\usepackage{caption} 
\captionsetup[table]{skip=3pt}  
\providecommand{\pr}[1]{\ensuremath{\Pr\left(#1\right)}}
\providecommand{\cbrak}[1]{\ensuremath{\left\{#1\right\}}}
\renewcommand{\thefigure}{\arabic{table}}
\renewcommand{\thetable}{\arabic{table}}                                     
                               
\title{AI1110 \\ Assignment 6}
\author{Pranav B \\ AI21BTECH11023}
\date{26 May 2022}


\begin{document}
	% The title page
	\begin{frame}
		\titlepage
	\end{frame}
	
	% The table of contents
	\begin{frame}{Outline}
    		\tableofcontents
	\end{frame}
	
	% The question
	\section{Question}
	\begin{frame}{ Papoulis Chapter 4 problem 4.34}
	We place at random n particles in $m > n$ boxes. Find the probability $p$ that the particles will be found in $n$ preselected boxes (one in each box). Consider the following cases:\\
(a) M-B (Maxwell-Boltzmann)-the particles are distinct; all alternatives are possible.\\
(b) B-E (Bose-Einstein)-the particles cannot be distinguished; all alternatives are possible.\\
(c) F-D (Fermi-Dirac )-the particles cannot be distinguished; at most one particle is allowed in a box.\\
	\end{frame}
	% The solution
	\section{Solution}
	\begin{frame}{Solution}
	a) $because$ all particles are distinct each particle can be put into any of the $m$ boxes.\\
	 $therefore$ total number of ways =$m^n$\\
	 Now the number of ways in which the balls can be present in the preselected boxes is = $n!$\\
	 $therefore p= \frac{n!}{m^n}$
	 \end{frame}
	 \begin{frame}
	 b)Given all particles are identical. Assume $k_i$ number of balls to be present in the ith box.\\
	 \begin{align}
	 \sum_{i=1}^{m}k_{i}=n
	 \end{align}
	 Number of possible solutions for this is = $\comb{m+n-1}{m-1}$\\
	 The number of favourable outcomes is =1\\
	 \begin{align}
	  \therefore p=\frac{1}{\comb{m+n-1}{m-1}}\\
	  \implies p=\frac{n!(m-1)!}{(m+n-1)!}
	 \end{align}
	 \end{frame}
	 \begin{frame}
	c)Since the particles are not distinguishable, Total ways equals the number of ways of selecting n out of m =$\comb{m}{n}$\\
	and favourable outcomes=1\\
	\begin{align}
	p=\frac{1}{\comb{m}{n}}\\
	\implies p=\frac{n!(m-n)!}{m!}
	\end{align}
	\end{frame}
\end{document}