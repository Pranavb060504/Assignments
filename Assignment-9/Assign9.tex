\documentclass{beamer}
\usetheme{CambridgeUS}

\setbeamertemplate{caption}[numbered]{}

\usepackage{enumitem}
\usepackage{tfrupee}
\usepackage{amsmath}
\usepackage{amssymb}
\usepackage{gensymb}
\usepackage{graphicx}
\usepackage{txfonts}

\def\inputGnumericTable{}
\newcommand*{\permcomb}[4][0mu]{{{}^{#3}\mkern#1#2_{#4}}}
\newcommand*{\perm}[1][-3mu]{\permcomb[#1]{P}}
\newcommand*{\comb}[1][-1mu]{\permcomb[#1]{C}}
\usepackage[latin1]{inputenc}                                 
\usepackage{color}                                            
\usepackage{array}                                            
\usepackage{longtable}                                        
\usepackage{calc}                                             
\usepackage{multirow}                                         
\usepackage{hhline}                                           
\usepackage{ifthen}
\usepackage{caption} 
\captionsetup[table]{skip=3pt}  
\providecommand{\pr}[1]{\ensuremath{\Pr\left(#1\right)}}
\providecommand{\cbrak}[1]{\ensuremath{\left\{#1\right\}}}
\renewcommand{\thefigure}{\arabic{table}}
\renewcommand{\thetable}{\arabic{table}}                                     
                               
\title{AI1110 \\ Assignment 9}
\author{Pranav B \\ AI21BTECH11023}
\date{ 12 June 2022}


\begin{document}
	% The title page
	\begin{frame}
		\titlepage
	\end{frame}
	
	% The table of contents
	\begin{frame}{Outline}
    		\tableofcontents
	\end{frame}
	
	% The question
	\section{Question}
	\begin{frame}{ Papoulis Chapter 9 problem 9.9}
 Show that the process $x(t)=c\omega(t)$ is WSS iff $E(c)=0$ and $\omega(t)=e^{j(\omega t+\theta)}$
	\end{frame}
	\section{Solution}
	\begin{frame}{Solution}
	 A stochastic process $x(t)$ is called wide-sense stationary if its mean is constant and $R_{x}(t_{1},t_{2})=R_{x}(T)$ , where $T=t_{1}-t_{2}$.\\
	 \begin{align}
	 \eta_{c}&=0\\
	 \implies  \eta_{x(t)}&=\eta_{c}e^{j(\omega t+\theta)}=0\\
	 R_{xx}(t+\tau,t)&={\sigma_{c}}^{2}e^{j \omega \tau} 
	 \end{align}
	 Hence $x(t)$ is WSS, Now lets prove the converse:\\
	\end{frame}
	\begin{frame}
	$\eta_{x(t)}&=\eta_{c} w(t)$ is independent of t;hence$\eta_{c}=0$.The function $R_{x}(t_{1},t_{2})={\sigma_{c}}^{2} w(t_1)w^{*}(t_2)$ depends only on $\tau=t_{1}-t_{2}$;Hence $\omega(t+\tau)\omega^{*}(t)=g(\tau)$, with $\tau=0$ this yields\\
	\begin{align}
	\left|\omega(t)\right|^{2}=g(0)=\text{constant}\\
	\omega(t+\tau)\omega^{*}(t)=a^{2}e^{j[\phi(t+\tau)-\phi(t)]}
	\end{align}
	\end{frame}
	\begin{frame}
	Hence the difference $\phi(t+\tau)-\phi(t)$ depends only on $\tau$\\
	From this it follows that,if $\phi(t)$ is continuous then,$\phi(t)$ is a linear function of $t$. Assuming $\phi(t)$ to be differentiable we get\\
	\begin{align}
	\phi'(t+\tau)=\phi'(t)\\
	\text{putting t=0}\\
	\phi'(\tau)=\phi'(0)=\text{constant}
	\end{align}
	\end{frame}
\end{document}