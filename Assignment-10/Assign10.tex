\documentclass{beamer}
\usetheme{CambridgeUS}

\setbeamertemplate{caption}[numbered]{}

\usepackage{enumitem}
\usepackage{tfrupee}
\usepackage{amsmath}
\usepackage{amssymb}
\usepackage{gensymb}
\usepackage{graphicx}
\usepackage{txfonts}

\def\inputGnumericTable{}
\newcommand*{\permcomb}[4][0mu]{{{}^{#3}\mkern#1#2_{#4}}}
\newcommand*{\perm}[1][-3mu]{\permcomb[#1]{P}}
\newcommand*{\comb}[1][-1mu]{\permcomb[#1]{C}}
\usepackage[latin1]{inputenc}                                 
\usepackage{color}                                            
\usepackage{array}                                            
\usepackage{longtable}                                        
\usepackage{calc}                                             
\usepackage{multirow}                                         
\usepackage{hhline}                                           
\usepackage{ifthen}
\usepackage{caption} 
\captionsetup[table]{skip=3pt}  
\providecommand{\pr}[1]{\ensuremath{\Pr\left(#1\right)}}
\providecommand{\cbrak}[1]{\ensuremath{\left\{#1\right\}}}
\renewcommand{\thefigure}{\arabic{table}}
\renewcommand{\thetable}{\arabic{table}}                                     
                               
\title{AI1110 \\ Assignment 10}
\author{Pranav B \\ AI21BTECH11023}
\date{ 12 June 2022}


\begin{document}
	% The title page
	\begin{frame}
		\titlepage
	\end{frame}
	
	% The table of contents
	\begin{frame}{Outline}
    		\tableofcontents
	\end{frame}
	
	% The question
	\section{Question}
	\begin{frame}{ Papoulis Chapter 9 problem 9.35}
Show that if $y(t)=x(t+a)-x(t-a)$ then,\\
$R_{y}(\tau)=2R_{x}(\tau)-R_{x}(\tau+2a)-R_{x}(\tau-2a)$\\
$S_{y}(\omega)=4S_{x}(\omega)\sin^{2}(a\omega)$
	\end{frame}
	\section{Solution}
	\begin{frame}{Solution}
	The process $y(t)=x(t+a)-x(t-a)$ is the output of a system with input $x(t)$ and system function\\
	$H(\omega)=e^{ja\omega}+e^{-ja\omega}=2j\sin(a\omega)$\\
	From corollary of convolution theorem,\\
	\begin{align}
	S_{yy}(\omega)=S_{xx}(\omega)H(\omega)H^{*}(\omega)\\
	\implies S_{yy}(\omega)=S_{xx}(\omega)|H(\omega)|^{2}\\
	\therefore S_{yy}(\omega)=S_{xx}(\omega)(2\sin^{2}(a\omega))\\
	\implies S_{y}(\omega)=4S_{x}(\omega)\sin^{2}(a\omega)\\
	\implies S_{y}(\omega)=(2-e^{2ja\omega}-e^{-2ja\omega})S_{x}(\omega)\\
	\therefore R_{y}(\tau)=2R_{x}(\tau)-R_{x}(\tau+2a)-R_{x}(\tau-2a)
	\end{align}
	\end{frame}
\end{document}